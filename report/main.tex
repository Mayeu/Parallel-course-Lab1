%%%%%%%%%%%%%%%%%%%%%%%%%%%%%%%%%%%%%%%%%%%%%%%%%%%%%%%%
%%%%%%                                            %%%%%%
%%%                                                  %%%
%      Modèle de Rapport.                              %
%               Par Matthieu Maury                     %
%                                                      %
%%%                                                  %%%
%%%%%%                                            %%%%%%
%%%%%%%%%%%%%%%%%%%%%%%%%%%%%%%%%%%%%%%%%%%%%%%%%%%%%%%%

\documentclass[10pt, a4paper]{report}

%%%%%%%%%%%%%%%%%%%%%%%%%%%%%%%%%%%%%%%%%%%%%%%%%%%%%%%%
%% Package essentiel
\usepackage[greek,english]{babel}
\usepackage[T1]{fontenc}
\usepackage{ucs}
\usepackage[utf8x]{inputenc}
\usepackage[top=2.6cm,bottom=2.6cm,right=2.1cm,left=2.1cm]{geometry}
\usepackage{float}

%%%%%%%%%%%%%%%%%%%%%%%%%%%%%%%%%%%%%%%%%%%%%%%%%%%%%%%%
%% Package optionnel
\usepackage{enumerate}
\usepackage{graphicx}
\usepackage{tabularx}
\usepackage{setspace}
\usepackage[dvips]{pstricks}
\usepackage{pstricks-add}
\usepackage{color}
\usepackage{xcolor}
\usepackage{epsfig}
\usepackage{pst-grad} % For gradients
\usepackage{pst-plot} % For axes
\usepackage{amsmath}
\usepackage{amsfonts}
\usepackage{amssymb}
\usepackage{amsxtra}
\usepackage{mathrsfs}
\usepackage{framed}
%\usepackage[framed, thmmarks, amsmath]{ntheorem}
\usepackage{verbatim}
\usepackage{moreverb}
\usepackage{fancyhdr}
\usepackage{url}
\usepackage{listings}
%\usepackage{hyperlinks}
\usepackage{lettrine}

%%%%%%%%%%%%%%%%%%%%%%%%%%%%%%%%%%%%%%%%%%%%%%%%%%%%%%%%
%%%%%%                                            %%%%%%
%%         Configuration de la mise en page           %%
%%%%%                                              %%%%%
%%%%%%%%%%%%%%%%%%%%%%%%%%%%%%%%%%%%%%%%%%%%%%%%%%%%%%%%

%%%%%%%%%%%%%%%%%%%%%%%%%%%%%%%%%%%%%%%%%%%%%%%%%%%%%%%%
%% Profondeur du sommaire
\setcounter{secnumdepth}{4}
\setcounter{tocdepth}{4}

%%%%%%%%%%%%%%%%%%%%%%%%%%%%%%%%%%%%%%%%%%%%%%%%%%%%%%%%
%% Configuration des chapitres
\makeatletter
\def\@makechapterhead#1{%
  \vspace*{50\p@}%
  {\parindent \z@ \raggedright \normalfont
    \interlinepenalty\@M
    \Huge \bfseries\thechapter.\quad#1\par\nobreak
    \vskip 20\p@
  }}
\makeatother

%%%%%%%%%%%%%%%%%%%%%%%%%%%%%%%%%%%%%%%%%%%%%%%%%%%%%%%%
%%%%%%                                            %%%%%%
%%                Début du Document                   %%
%%%%%                                              %%%%%
%%%%%%%%%%%%%%%%%%%%%%%%%%%%%%%%%%%%%%%%%%%%%%%%%%%%%%%%


\lstset{tabsize=3, inputencoding=utf8x, extendedchars=\true, language=C}

\input{my_math}

\begin{document}

%%%%%%%%%%%%%%%%%%%%%%%%%%%%%%%%%%%%%%%%%%%%%%%%%%%%%%%%
%% Inclusion de la page de titre
\pagestyle{fancy}
\renewcommand{\sectionmark}[1]{\markright{\thesection\ #1}}
\renewcommand{\footrulewidth}{0pt}
\renewcommand{\headrulewidth}{0pt}
\fancyhead{} % clear all header fields
\fancyfoot{} % clear all footer fields
\fancyfoot[LO,RE]{\textit{University year 2009-2010}}
\fancyfoot[LE,RO]{\textit{Written with \LaTeX}}

\begin{tabularx}{17cm}{Xr}
  \begin{tabular}{ll}
    Yohann Teston & 881003-P792\\
    \url{yohann.teston@free.fr} &\\
	Matthieu Maury & 860928-P210\\
	\url{mayeu.tik@gmail.com} & \\
  \end{tabular} 

  &
  
  \begin{tabular}{r}
    \includegraphics[width=5cm]{pic/logoupp.eps} \\
    \textit{Department of Information Technology} \\
  \end{tabular}
\end{tabularx}

\vspace{6cm}

\begin{center}
  \textbf{ {\Huge Programming of parallel computers}}\\[0.5em]{\huge Lab 1 - MPI}
\end{center}

\begin{center}
  \today
\end{center}


\newpage


\thispagestyle{empty}
%\input{resume}

\renewcommand{\footrulewidth}{0.5pt}
\renewcommand{\headrulewidth}{0.5pt}
\fancyhead{} % clear all header fields
\fancyhead[RE,LO]{Programming of parallel computers - Lab 1 - MPI}


\fancyhead[RO,LE]{\rightmark}

\fancyfoot{} % clear all footer fields
\fancyfoot[LO,RE]{Matthieu Maury \& Yohann Teston}
\fancyfoot[LE,RO]{\thepage}

%Redéfinition du style fancy - plain, utilisé pour les pages de nouveau chapitre
%Le style par défaut est un style plain
\fancypagestyle{plain}{
    \fancyhf{}
    \renewcommand{\headrulewidth}{0pt}

    %Définition des headers identiques à une page normale
    \fancyfoot[LO,RE]{Matthieu Maury \& Yohann Teston}
    \fancyfoot[LE,RO]{\thepage}
}

\tableofcontents

%\listoffigures

%\newpage

%\doublespacing
\onehalfspacing
\input{q1}
\chapter{Point-to-point communication}

\section{Work on \textit{exchange.c}}

The main advantage of using non-blocking calls is that it becomes possible to overlap communication and computation. Indeed, those calls start the process of sending or receiving and return without completing it. The caller is then free to do some other computation during the communication and is not blocked until the end of the communication, as it is the case with a blocking call. The main drawback is that the programmer must make sure not to alter the sending/receiving buffer because they are being used even after the non-blocking call has returned. This makes programming harder.

\section{Work on \textit{pingpong.c}}

By tracing the data we obtain:

%\begin{figure}[h]
  \begin{center}
	 \psfig{figure=pic/ex2.eps}
  \end{center}
%  \caption{}
%  \label{fig:}
%\end{figure}

The average latency is: $6.8794us$ and the average bandwidth: $522.1591 MB.s^{-1}$.

The graph is linear, showing that the latency between sending and receiving a message is constant during the whole send/recv program.


\chapter{Collective communication, global data}

\section{Pass-on}

As can be seen in the following picture, the \textit{pass-on} method has been successfully implemented. For a better understanding of what is going on, the messages have been labelled with the rank of the sender. Thus, we can see that the value is correctly passed from a processor to its direct neighbor.
\begin{figure}[!h]
\begin{center}
	\includegraphics[width=\textwidth]{pic/passon.eps}
	\caption{Pass-on}
\end{center}
\end{figure}

\section{Fan-out}

Like the last version we print the rank of the sender. In this version we set the number of steps with:
$$ceil(ln(NP))$$
where $NP$ is the total number of processors.

At each step we define the senders as:
$$S_{step}=\{\sqqs p, rank < 2^{step} \sand rank + 2^{step} < NP\}$$
where $p$ is a processor. This disallows processors which did not receive data to send, and processors which will send data to a non-existant processor.

Following the same principle we define the receivers as:
$$R_{step}=\{\sqqs p, rank < 2^{step+1} \sand rank - 2^{step} \ge 0\}$$
by using this, only processors that will receive a message at this step will fetch the message.\\

There is the output of our implementation:

\begin{verbatim}
Lain-ux@nyarlathothep:code > mpirun -np 8 a.out 
Processor 1 got 999.999000 from 0
Processor 2 got 999.999000 from 0
Processor 3 got 999.999000 from 1
Processor 4 got 999.999000 from 0
Processor 5 got 999.999000 from 1
Processor 6 got 999.999000 from 2
Processor 7 got 999.999000 from 3
\end{verbatim}

As we can see, not everybody sends messages.

\section{Broadcast}

The following picture allows us to verify the correctness of the program using \textit{MPI\_Bcast}.
\begin{figure}[!h]
\begin{center}
	\includegraphics[width=\textwidth]{pic/bcast.eps}
	\caption{Broadcast}
\end{center}
\end{figure}

MPI does have a lot of other \textit{collective communication} operations, described as follows:

\begin{itemize}
	\item Barrier synchronization across all group members
	\item Broadcast from one member to all members of a group
	\item Gather data from all group members to one member 
	\item Scatter data from one member to all members of a group
	\item A variation on Gather where all members of the group receive the result
	\item Scatter/Gather data from all members to all members of a group
	\item Global reduction operations such as sum, max, min, or user-defined functions, where the result is returned to all group members and a variation where the result is returned to only one member (used in the following exercise) 
	\item A combined reduction and scatter operation
	\item Scan across all members of a group (also called prefix)
\end{itemize}

\input{q4}
\input{q5}
\input{q6}
\input{q7}

%\newpage
%\setcounter{page}{1}
%\pagenumbering{Roman}
%\appendix
%\chapter{Compile and run}

\lstinputlisting{../code/hello.c}

\chapter{Point-to-point}

\section{Exchange.c}
\lstinputlisting{../code/exchange.c}

\section{Ping pong}
\lstinputlisting{../code/pingpong.c}

\chapter{Collective communication}

\section{pass-on}
\lstinputlisting{../code/onetoall.c}

\section{fan-out}
\lstinputlisting{../code/fanout.c}

\section{onetoall\_glob}
\lstinputlisting{../code/onetoall_glob.c}

\chapter{Reduction}

\lstinputlisting{../code/pi.c}

\chapter{Derived datatype}

\lstinputlisting{../code/datatypes.c}

\chapter{Communicators}

\lstinputlisting{../code/communicators.c}


\end{document}
